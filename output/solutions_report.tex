
\documentclass[10pt]{article}
\usepackage[utf8]{inputenc}
\usepackage{amsmath}
\usepackage{amssymb}
\usepackage{geometry}
\usepackage{xcolor}
\usepackage{longtable}
\usepackage{array}
\usepackage{lmodern}

\geometry{a4paper, margin=1in}

\definecolor{correct}{HTML}{28a745}
\definecolor{partial}{HTML}{FFC107}
\definecolor{incorrect}{HTML}{DC3545}
\definecolor{noerror}{HTML}{6C757D}

\title{LLM Benchmark Report}
\author{Generated by script}
\date{\today}

\begin{document}
\maketitle

\section*{Summary Statistics}
\begin{center}
\begin{longtable}{|l|c|c|c|c|c|c|}
\hline
\textbf{Model} & \textbf{Total} & \textbf{Correct} & \textbf{Partial} & \textbf{Incorrect} & \textbf{Errors} & \textbf{Avg. Score} \\
\hline
\endfirsthead
\hline
\endfoot
o3 & 5 & 5 & 0 & 0 & 0 & 1.00 \\
\hline
\end{longtable}
\end{center}
\section*{Problem Details}

\subsection*{Problem 1 (Paper: 2506.18771v1)}
\subsubsection*{Problem Statement}
## Problem

Background:
The flow behavior of granular materials in varying gravitational environments is characterized by the dimensionless granular Bond number, $\mathcal{B}$. This number quantifies the ratio of flow-inhibiting cohesive forces to flow-inducing gravitational forces. For a granular bed with total sample mass $m$ and cross-sectional area $A$, subjected to gravitational acceleration $g$, the granular Bond number is defined as $\mathcal{B}(g) = \frac{\sigma_\text{t}}{mg/A}$. Here, $\sigma_\text{t}$ represents the granular tensile strength, which is an intrinsic material property and is considered independent of the gravitational acceleration. During fluidization experiments, the steady-state pressure drop across the granular bed, $\Delta p_\text{SS}(g)$, is observed to balance the particles' weight per unit area, such that $\Delta p_\text{SS}(g) = mg/A$. We denote Earth's gravitational acceleration as $g_E$ and the Moon's gravitational acceleration as $g_L$.

Task:
Derive an analytical expression for the granular tensile strength, $\sigma_\text{t}$, in terms of the granular Bond number at Moon gravity, $\mathcal{B}(g_L)$, the gravitational acceleration on the Moon, $g_L$, and the steady-state pressure drop during fluidization at Earth gravity, $\Delta p_\text{SS}(g_E)$.

###

\subsubsection*{Ground Truth Solution}
\[ \boxed{\sigma_\text{t} = \mathcal{B}(g_L) \frac{\Delta p_\text{SS}(g_E)}{g_E} g_L} \]

\subsubsection*{Model Outputs}
\subsubsection*{Model: o3 (Score: 1.0)}
\paragraph*{Model Solution:}
\;
      \sigma_{t}= \mathcal{B}(g_L)\,\frac{g_L}{g_E}\,
      \Delta p_{\text{SS}}(g_E)
      \;

\paragraph*{Judge's Evaluation:}
\begin{quote}\ttfamily Step-by-step reasoning:

1. The ground truth expression is:
   \[
   \sigma\_\text\{t\} = \mathcal\{B\}(g\_L) \frac\{\Delta p\_\text\{SS\}(g\_E)\}\{g\_E\} g\_L
   \]
   This can be rewritten as:
   \[
   \sigma\_\text\{t\} = \mathcal\{B\}(g\_L) \left(\frac\{g\_L\}\{g\_E\}\right) \Delta p\_\text\{SS\}(g\_E)
   \]

2. The model-generated expression is:
   \[
   \sigma\_t = \mathcal\{B\}(g\_L) \frac\{g\_L\}\{g\_E\} \Delta p\_\{\text\{SS\}\}(g\_E)
   \]

3. Comparing the two, the model-generated expression matches the ground truth exactly, just with a slightly different order of terms, which is mathematically equivalent.

4. There are no missing variables, coefficients, or sign errors. The expression is fully consistent with the ground truth.

Therefore, the model-generated expression is completely correct.

Score: 1\end{quote}

\newpage
\subsection*{Problem 2 (Paper: 2506.18821v1)}
\subsubsection*{Problem Statement}
## Problem

Background:
Consider a coarse-grained model of a double-stranded DNA molecule consisting of $N$ beads, where each bead $i$ is located at position $\vec{r}_i$. The total potential energy of the system, $E$, is given by:
$$ E = \sum_{i=1}^{N-1}K(d_{i,i+1}-d_0)^2 + 4 \sum_{\rm Non-native}\bigg(\frac{\sigma_{i,j}}{d_{i,j}}\bigg)^{12} +4{\epsilon^{HB}_{i,j}}\sum_{\rm Native-contacts} \bigg[\bigg(\frac{\sigma_{i,j}}{d_{i,j}}\bigg)^{12} -\bigg(\frac{\sigma_{i,j}}{d_{i,j}}\bigg)^6\bigg] $$
Here, $K$ is the spring constant, $d_0$ is the equilibrium bond distance, and $d_{i,j} = |\vec{r}_i - \vec{r}_j|$ is the distance between beads $i$ and $j$. The parameter $\sigma_{i,j}$ is a distance scale for interactions, and $\epsilon^{HB}_{i,j}$ represents the hydrogen bonding interaction energy between base-pairs. The first term represents harmonic bonds between consecutive beads. The second term sums over "Non-native" pairs $(i,j)$, defined as pairs where $j \ne i \pm 1$ and $j \ne N-i+1$. The third term sums over "Native-contacts", defined as pairs $(i, N-i+1)$, representing hydrogen bonding between complementary base-pairs. An external pulling force $f$ is applied to bead $N$ along the positive y-axis, while bead $1$ is fixed at the origin. This external force adds an energy term $-f r_{N,y}$ to the total energy, where $r_{N,y}$ is the y-coordinate of bead $N$. The total force $\mathbf{F}_k$ acting on any bead $k$ is given by $\mathbf{F}_k = -\nabla_k E_{total}$, where $E_{total} = E - f r_{N,y}$. Let $\hat{u}_{i,j} = (\vec{r}_i - \vec{r}_j) / d_{i,j}$ denote the unit vector pointing from bead $j$ to bead $i$.

Task:
Derive the total force $\mathbf{F}_k$ acting on an arbitrary bead $k$ (where $1 < k < N$), expressing it as a sum of contributions from bonded, non-native, and native interactions.

###

\subsubsection*{Ground Truth Solution}
\[ \boxed{\begin{aligned}
\mathbf{F}_k &= \mathbf{F}_{k, \text{bond}} + \mathbf{F}_{k, \text{non-native}} + \mathbf{F}_{k, \text{native}} \\
\mathbf{F}_{k, \text{bond}} &= 2K(d_{k-1,k}-d_0) \hat{u}_{k-1,k} + 2K(d_{k,k+1}-d_0) \hat{u}_{k+1,k} \\
\mathbf{F}_{k, \text{non-native}} &= 48 \sum_{j \in \{1, \dots, N\} \setminus \{k, k-1, k+1, N-k+1\}} \sigma_{k,j}^{12} d_{k,j}^{-13} \hat{u}_{k,j} \\
\mathbf{F}_{k, \text{native}} &= 4{\epsilon^{HB}_{k, N-k+1}} \left[ 12 \sigma_{k, N-k+1}^{12} d_{k, N-k+1}^{-13} - 6 \sigma_{k, N-k+1}^6 d_{k, N-k+1}^{-7} \right] \hat{u}_{k, N-k+1}
\end{aligned}} \]

\subsubsection*{Model Outputs}
\subsubsection*{Model: o3 (Score: 1.0)}
\paragraph*{Model Solution:}
\;
\begin{aligned}
\mathbf F_k &=
      2K\bigl(d_{k-1,k}-d_0\bigr)\hat u_{k-1,k}
     -2K\bigl(d_{k,k+1}-d_0\bigr)\hat u_{k,k+1} \\[4pt]
    &\quad+48\sum_{\substack{j=1\\ j\neq k\pm1,\,j\neq N-k+1}}^{N}
          \frac{\sigma_{k,j}^{\,12}}{d_{k,j}^{\,13}}\,
          \hat u_{k,j} \\[4pt]
    &\quad+\Bigl[
            48\epsilon^{HB}_{k,j^*}\frac{\sigma_{k,j^*}^{\,12}}{d_{k,j^*}^{\,13}}
           -24\epsilon^{HB}_{k,j^*}\frac{\sigma_{k,j^*}^{\,6}}{d_{k,j^*}^{\,7}}
          \Bigr]\hat u_{k,j^*}\;,
\end{aligned}\;

\paragraph*{Judge's Evaluation:}
\begin{quote}\ttfamily Let's analyze the model-generated expression term-by-term and compare it with the ground truth.

---

\#\#\# 1. Bonded Force Term

**Ground Truth:**
\[
\mathbf\{F\}\_\{k, \text\{bond\}\} = 2K(d\_\{k-1,k\} - d\_0) \hat\{u\}\_\{k-1,k\} + 2K(d\_\{k,k+1\} - d\_0) \hat\{u\}\_\{k+1,k\}
\]

**Model:**
\[
2K(d\_\{k-1,k\} - d\_0) \hat\{u\}\_\{k-1,k\} - 2K(d\_\{k,k+1\} - d\_0) \hat\{u\}\_\{k,k+1\}
\]

**Comparison:**

- The first term matches exactly.
- The second term differs in two ways:
  - The sign is negative in the model, but positive in the ground truth.
  - The unit vector in the ground truth is \(\hat\{u\}\_\{k+1,k\} = \frac\{\vec\{r\}\_\{k+1\} - \vec\{r\}\_k\}\{d\_\{k,k+1\}\}\), pointing from bead \(k\) to \(k+1\) but reversed (from \(k+1\) to \(k\)).
  - The model uses \(\hat\{u\}\_\{k,k+1\} = \frac\{\vec\{r\}\_k - \vec\{r\}\_\{k+1\}\}\{d\_\{k,k+1\}\}\), which is the negative of \(\hat\{u\}\_\{k+1,k\}\).

Because \(\hat\{u\}\_\{k,k+1\} = -\hat\{u\}\_\{k+1,k\}\), the model's negative sign combined with \(\hat\{u\}\_\{k,k+1\}\) is equivalent to a positive sign with \(\hat\{u\}\_\{k+1,k\}\).

Hence, the model's second term is actually equivalent to the ground truth second term.

**Conclusion:** Bonded force terms are equivalent.

---

\#\#\# 2. Non-native Force Term

**Ground Truth:**
\[
48 \sum\_\{j \in \\{1, \dots, N\\} \setminus \\{k, k-1, k+1, N-k+1\\}\} \sigma\_\{k,j\}^\{12\} d\_\{k,j\}^\{-13\} \hat\{u\}\_\{k,j\}
\]

**Model:**
\[
48 \sum\_\{\substack\{j=1 \\ j \neq k \pm 1, j \neq N-k+1\}\}^\{N\} \frac\{\sigma\_\{k,j\}^\{12\}\}\{d\_\{k,j\}^\{13\}\} \hat\{u\}\_\{k,j\}
\]

**Comparison:**

- The summation indices exclude \(j = k \pm 1\) and \(j = N-k+1\), matching the ground truth.
- The model does not explicitly exclude \(j = k\), but since \(j\) runs from 1 to \(N\) and the term \(\sigma\_\{k,k\}\) would be meaningless (distance zero), it is implicitly excluded or zero.
- The factor 48 and the powers match exactly.
- The unit vector \(\hat\{u\}\_\{k,j\}\) matches.

**Conclusion:** Non-native force term matches the ground truth.

---

\#\#\# 3. Native Force Term

**Ground Truth:**
\[
4 \epsilon^\{HB\}\_\{k, N-k+1\} \left[ 12 \sigma\_\{k, N-k+1\}^\{12\} d\_\{k, N-k+1\}^\{-13\} - 6 \sigma\_\{k, N-k+1\}^6 d\_\{k, N-k+1\}^\{-7\} \right] \hat\{u\}\_\{k, N-k+1\}
\]

**Model:**
\[
\left[ 48 \epsilon^\{HB\}\_\{k,j^*\} \frac\{\sigma\_\{k,j^*\}^\{12\}\}\{d\_\{k,j^*\}^\{13\}\} - 24 \epsilon^\{HB\}\_\{k,j^*\} \frac\{\sigma\_\{k,j^*\}^6\}\{d\_\{k,j^*\}^7\} \right] \hat\{u\}\_\{k,j^*\}
\]

where \(j^* = N-k+1\).

**Comparison:**

- The model uses \(j^*\) to denote the native contact partner, which is consistent.
- The coefficients inside the bracket differ by a factor of 4:
  - Ground truth: \(4 \times [12 \cdots - 6 \cdots] = 48 \cdots - 24 \cdots\)
  - Model: directly writes \(48 \cdots - 24 \cdots\)
- So the model's expression is algebraically equivalent to the ground truth.
- The unit vector \(\hat\{u\}\_\{k,j^*\}\) matches.
- The powers and terms match exactly.

**Conclusion:** Native force term matches the ground truth.

---

\#\#\# 4. External Force Term

- The problem states the total energy includes \(-f r\_\{N,y\}\), so the force on bead \(k\) is \(-\nabla\_k E\_\{total\}\).
- The ground truth expression does not explicitly include the external force term because \(k \neq N\), and the external force acts only on bead \(N\).
- The model expression also does not include any external force term for bead \(k\), which is correct.

---

\#\#\# 5. Summary

- Bonded terms: equivalent (sign and unit vector differences cancel out).
- Non-native terms: match exactly.
- Native terms: match exactly (coefficients consistent).
- External force: correctly omitted for \(k \neq N\).

No fundamental errors found.

---

**Final assessment:** The model-generated expression is mathematically equivalent to the ground truth expression, with only minor cosmetic differences in notation and summation indices.

**Score: 1**\end{quote}

\newpage
\subsection*{Problem 3 (Paper: 2506.18823v1)}
\subsubsection*{Problem Statement}
## Problem

Background:
The dynamics of a reaction coordinate $q(t)$ in an overdamped system can be described by a Langevin equation with a coordinate-dependent friction coefficient $\gamma(q)$ and a free energy profile $U(q)$. The mean first-passage time (MFPT), denoted as $\tau_{\rm{MFP}}(q_S, q_F)$, is the average time it takes for the reaction coordinate to reach a final position $q_F$ for the first time, starting from an initial position $q_S$. For the case where $q_S < q_F$, the MFPT is given by the expression:
$$
\tau_{\text {MFP}}\left(q_S, q_F\right)=\beta \int_{q_S}^{q_F} \mathrm{~d} q e^{\beta U(q)} \gamma(q) \int_{q_{\min }}^q \mathrm{~d} q^{\prime} e^{-\beta U\left(q^{\prime}\right)}
$$
Here, $\beta = 1/(k_B T)$ is the inverse thermal energy, where $k_B$ is the Boltzmann constant and $T$ is the temperature. $q_{\min}$ represents the minimum possible value of the reaction coordinate. Let $Z_1$ be defined as $Z_1 = \int_{q_{\min }}^{q_F} \mathrm{~d} q^{\prime} e^{-\beta U\left(q^{\prime}\right)}$.

Task:
Derive an expression for the coordinate-dependent friction coefficient $\gamma(q_F)$ in terms of $\tau_{\rm{MFP}}(q_S, q_F)$, its partial derivative with respect to $q_F$, the free energy $U(q_F)$, and $Z_1$.

###

\subsubsection*{Ground Truth Solution}
\[ \boxed{\gamma(q_F) = k_B T \frac{e^{-\beta U(q_F)}}{Z_1} \frac{\partial \tau_{\mathrm{MFP}}}{\partial q_F}} \]

\subsubsection*{Model Outputs}
\subsubsection*{Model: o3 (Score: 1.0)}
\paragraph*{Model Solution:}
\;
\gamma(q_F)=\frac{e^{-\beta U(q_F)}}{\beta\,Z_1(q_F)}
\;\frac{\partial \tau_{\mathrm{MFP}}(q_S,q_F)}{\partial q_F}
\;

\paragraph*{Judge's Evaluation:}
\begin{quote}\ttfamily Step-by-step reasoning:

1. The ground truth expression is:
   \[
   \gamma(q\_F) = k\_B T \frac\{e^\{-\beta U(q\_F)\}\}\{Z\_1\} \frac\{\partial \tau\_\{\mathrm\{MFP\}\}\}\{\partial q\_F\}
   \]

2. The model-generated expression is:
   \[
   \gamma(q\_F) = \frac\{e^\{-\beta U(q\_F)\}\}\{\beta Z\_1(q\_F)\} \frac\{\partial \tau\_\{\mathrm\{MFP\}\}(q\_S,q\_F)\}\{\partial q\_F\}
   \]

3. Recall that \(\beta = \frac\{1\}\{k\_B T\}\). Therefore,
   \[
   \frac\{1\}\{\beta\} = k\_B T
   \]

4. Substituting \(1/\beta = k\_B T\) into the model expression:
   \[
   \gamma(q\_F) = k\_B T \frac\{e^\{-\beta U(q\_F)\}\}\{Z\_1(q\_F)\} \frac\{\partial \tau\_\{\mathrm\{MFP\}\}\}\{\partial q\_F\}
   \]

5. The model expression matches the ground truth expression exactly, except that the model explicitly writes \(Z\_1(q\_F)\) instead of \(Z\_1\). Since \(Z\_1\) is defined as an integral up to \(q\_F\), writing \(Z\_1(q\_F)\) is consistent and not a difference in meaning.

6. The model expression includes the arguments \((q\_S, q\_F)\) in \(\tau\_\{\mathrm\{MFP\}\}\), which is consistent with the problem statement and does not change the meaning.

7. There are no sign errors, missing factors, or other discrepancies.

Conclusion: The model-generated expression is mathematically equivalent to the ground truth expression, with only minor cosmetic differences in notation.

Score: 1\end{quote}

\newpage
\subsection*{Problem 4 (Paper: 2506.18835v1)}
\subsubsection*{Problem Statement}
## Problem

Background:
Consider a system where a single linear polyethyleneimine (PEI) molecule forms a stable complex with $N_{Hg}$ divalent mercury ions (Hg$^{2+}$) in an aqueous environment. The sum of electronic and thermal free energies for this complex is denoted as $G(\text{PEI} + N_{Hg}\text{Hg}^{2+})$. For the isolated components, the sum of electronic and thermal free energies for a single PEI molecule in water is $G(\text{PEI})$, and for a single unbound Hg$^{2+}$ ion in water, it is $G(\text{Hg}^{2+})$. The total Gibbs free energy of complexation for the entire system, $G_{\text{cmpx, total}}$, represents the energy change associated with the formation of the complex from its isolated components.

Task:
Derive the mathematical expression for the average Gibbs free energy of complexation per mercury ion, denoted as $G_{\text{cmpx, avg}}$.

###

\subsubsection*{Ground Truth Solution}
\[ \boxed{G_{\text{cmpx, avg}} = \frac{G(\text{PEI} + N_{Hg}\text{Hg}^{2+}) - G(\text{PEI})}{N_{Hg}} - G(\text{Hg}^{2+})} \]

\subsubsection*{Model Outputs}
\subsubsection*{Model: o3 (Score: 1.0)}
\paragraph*{Model Solution:}
\,G_{\text{cmpx, avg}}
       = \frac{G\!\left(\text{PEI}+N_{Hg}\text{Hg}^{2+}\right)
               -G(\text{PEI})
               -N_{Hg}\,G(\text{Hg}^{2+})}
              {N_{Hg}}\,

\paragraph*{Judge's Evaluation:}
\begin{quote}\ttfamily Step-by-step reasoning:

1. The ground truth expression for the average Gibbs free energy of complexation per mercury ion is:
   \[
   G\_\{\text\{cmpx, avg\}\} = \frac\{G(\text\{PEI\} + N\_\{Hg\}\text\{Hg\}^\{2+\}) - G(\text\{PEI\})\}\{N\_\{Hg\}\} - G(\text\{Hg\}^\{2+\})
   \]
   This means:
   - First, find the total free energy change of the complex relative to the PEI alone.
   - Divide that total change by the number of mercury ions to get the average per ion.
   - Then subtract the free energy of a single unbound Hg\(^\{2+\}\) ion.

2. The model-generated expression is:
   \[
   G\_\{\text\{cmpx, avg\}\} = \frac\{G(\text\{PEI\} + N\_\{Hg\}\text\{Hg\}^\{2+\}) - G(\text\{PEI\}) - N\_\{Hg\} G(\text\{Hg\}^\{2+\})\}\{N\_\{Hg\}\}
   \]
   This means:
   - The numerator is the total free energy of the complex minus the free energy of PEI minus the total free energy of all \(N\_\{Hg\}\) mercury ions.
   - Then the entire difference is divided by \(N\_\{Hg\}\).

3. Let's check if these two expressions are equivalent:

   Starting from the model expression:
   \[
   G\_\{\text\{cmpx, avg\}\} = \frac\{G(\text\{PEI\} + N\_\{Hg\}\text\{Hg\}^\{2+\}) - G(\text\{PEI\}) - N\_\{Hg\} G(\text\{Hg\}^\{2+\})\}\{N\_\{Hg\}\} = \frac\{G(\text\{PEI\} + N\_\{Hg\}\text\{Hg\}^\{2+\}) - G(\text\{PEI\})\}\{N\_\{Hg\}\} - G(\text\{Hg\}^\{2+\})
   \]

   This is exactly the ground truth expression.

4. Therefore, the model-generated expression is mathematically equivalent to the ground truth expression, just written in a different but fully correct form.

5. There are no missing variables, no wrong signs, and no miscalculations.

Conclusion: The model-generated expression is completely correct.

Score: 1\end{quote}

\newpage
\subsection*{Problem 5 (Paper: 2506.19029v1)}
\subsubsection*{Problem Statement}
## Problem

### Background:
Consider a two-dimensional binary mixture maintained close to its critical point, where its dynamics can be described by an effective Ginzburg-Landau free energy. The system is subject to a smoothly varying temperature profile, which is characterized by a reduced temperature field $\tau(\x) = (T(\x)-T_c)/T_c$, where $T_c$ is the critical temperature. A circular phase-separated droplet of radius $R_0$ is trapped at the origin, but deforms due to an anisotropic temperature modulation. The temperature modulation is given by $\tau(\x) = \Delta T/T_c + (\delta T/T_c) [1 + \frac{1}{2}(m_x(x/R_0)^2 + m_y(y/R_0)^2)]$, where $\Delta T < 0$ represents the background temperature difference from $T_c$, and $\delta T$ is a small perturbation such that $|\delta T| \ll |\Delta T|$. The droplet's deformed radius is $R(\theta) = R_0 + \delta R(\theta)$, where $\theta$ is the polar angle and $\delta R(\theta)$ is a small perturbation.

In the stationary state, the interface equation for the droplet, linearized to first order in $\delta R$, $\delta T$, and $\delta \mu_{\infty}$ (the perturbation to the integration constant $\mu_{\infty}$), is given by:
$$
\dfrac{\sigma^*}{\Delta\bar{\phi}^*}\frac{\delta R''(\theta)+\delta R(\theta)}{R_0^2} - \frac{\delta T (m_x-m_y)}{2}\left(\Sigma_T\cos2\theta+\frac{1}{T_c}\right) + \delta\mu_{\infty}=0
$$
where $\sigma^*$ and $\Delta\bar{\phi}^*$ are the surface tension and binodal difference, respectively, evaluated at $\tau \approx \Delta T/T_c$, and $\Sigma_T$ is the Soret coefficient. The mass conservation condition for the droplet implies that the integral of the deformation over the entire angle vanishes: $\int_0^{2\pi}d\theta\,\delta R(\theta)=0$.

### Task:
Derive the expression for the relative deformation of the droplet, $\delta R(\theta)/R_0$.

###

\subsubsection*{Ground Truth Solution}
\[ \boxed{\frac{\delta R(\theta)}{R_0}= -\frac{\Sigma_T}{6}\frac{\Delta\bar{\phi}^* R_0}{\sigma^*}\delta T(m_x-m_y)\cos2\theta} \]

\subsubsection*{Model Outputs}
\subsubsection*{Model: o3 (Score: 1.0)}
\paragraph*{Model Solution:}
\,\displaystyle 
\frac{\delta R(\theta)}{R_{0}}
          =-\,\frac{\delta T\,(m_{x}-m_{y})\,\Sigma_T\,R_{0}}
                 {6}\,\frac{\Delta\bar{\phi}^{*}}{\sigma^{*}}
          \,\cos 2\theta\,

\paragraph*{Judge's Evaluation:}
\begin{quote}\ttfamily Step-by-step reasoning:

1. The ground truth expression is:
\[
\frac\{\delta R(\theta)\}\{R\_0\} = -\frac\{\Sigma\_T\}\{6\} \frac\{\Delta\bar\{\phi\}^* R\_0\}\{\sigma^*\} \delta T (m\_x - m\_y) \cos 2\theta
\]

2. The model-generated expression is:
\[
\frac\{\delta R(\theta)\}\{R\_0\} = - \frac\{\delta T (m\_x - m\_y) \Sigma\_T R\_0\}\{6\} \frac\{\Delta\bar\{\phi\}^*\}\{\sigma^*\} \cos 2\theta
\]

3. Comparing the two:
   - Both have the factor \(-1\).
   - Both have \(\delta T (m\_x - m\_y) \Sigma\_T\).
   - Both have \(R\_0\).
   - Both have a factor of \(1/6\).
   - Both have the ratio \(\frac\{\Delta\bar\{\phi\}^*\}\{\sigma^*\}\).
   - Both multiply by \(\cos 2\theta\).

4. The order of factors is different, but multiplication is commutative, so this is a cosmetic difference.

5. No variables or coefficients are missing or incorrect.

6. The expression is mathematically equivalent.

Therefore, the model-generated expression matches the ground truth expression exactly, modulo cosmetic differences.

Score: 1\end{quote}

\newpage

\end{document}
