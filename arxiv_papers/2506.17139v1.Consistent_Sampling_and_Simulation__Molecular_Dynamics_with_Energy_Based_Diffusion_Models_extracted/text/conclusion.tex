\section{Conclusion, Limitations and Future Work}
\label{sec:conclusion}
In this work, we investigated the gap between independent sampling and simulation using diffusion models. We introduced a Fokker-Planck-based regularization on the model's energy and showed that reducing the deviation from the Fokker-Planck equation improves the consistency of the model. Additionally, we demonstrated that restricting the model’s focus to smaller diffusion timescales further improves simulation quality. We validated these findings across multiple systems, from toy examples to realistic biomolecular systems. This improved consistency enables the same model to be used for sampling and simulation, giving access to both kinetic and static properties.

Despite the theoretical motivation behind our approach, the results presented are primarily empirical. While our results indicate that reducing the Fokker-Planck deviation improves consistency, we do not claim this to be the only source of error. In fact, due to the fundamental differences between diffusion sampling and Langevin simulation, perfect alignment may not be achievable without limiting model expressivity. Additionally, evaluating the Fokker-Planck residual introduces computational overhead, which we mitigate through a weak residual formulation, although it still requires multiple forward passes of the model during training.

Future work could explore applying this approach to larger molecular systems, including proteins. It may also be promising to fine-tune pre-trained models with the proposed regularization or to train an auxiliary model to correct the identified inconsistencies, leveraging the explicit residual formulation.
